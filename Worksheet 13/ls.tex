% Author: Katherine Shu, Albert Tang
\\ \\
\learning{
    Understand when Least Squares is helpful for estimating values, and how to translate a word problem with given data points into a Least Squares set up. 
}
\\ \\
\begin{enumerate}
    \item{Write Ohm's Law for a resistor.}
    
    \sol{
        For the resistor
        \begin{center}
            % not sure why circuitikz is giving an error
            %\begin{circuitikz}
                %\draw
                %(0,0) to[R,l=$R$,v=$V_R$, i=$I_R$] ++(2,0);
            %\end{circuitikz}
        \end{center}
        $$V_R = I_RR$$
    }

    \item{You're given the following test setup and told to find $R_{Th}$ between the two terminals of the mystery box. What is $R_{Th}$ of the mystery box between the two terminals in terms of $V_S$ and $I_\text{measure}$?}
    
    % not sure why circuitikz is giving an error
    %\begin{circuitikz} [baseline=(current bounding box.center)]
        %\ctikzset { label/align = straight }
        %\draw (0,0)
            %to[V=$V_{S}$,invert] (0,2)
            %to[short, i=$I_\text{measure}$, -o] (4,2)
            %to[short] (4.5,2)
            %(0,0) to[short,-o] (4,0)
            %to[short] (4.5,0);
        %\node[draw,minimum width=2cm,minimum height=2.4cm,anchor=south west] at (4.5,-0.2){Mystery Box};
    %\end{circuitikz}

    \sol{
        \begin{align*}
            R_{Th} &= \frac{V_S}{I_\text{measure}}\\
            R_{Th} &= \frac{V_S}{I_\text{measure}}
        \end{align*}
    }
    
    \item{
        You think you've figured out how to find $R_{Th}$! You've taken the following measurements:
        \begin{center}
            \begin{tabular}{|c|c|c|}
                \hline
                Measurement \# & $I_\text{measure}$& $V_S$\\\hline
                1 & $1A$ & $1.25kV$\\\hline
                2 & $2A$ & $1kV$\\\hline
                3 & $3A$ & $4kV$\\\hline
                4 & $4A$ & $3.5kV$
            \end{tabular}
        \end{center}
        Using the information above, formulate a least squares problem whose answer provides an estimate of $R_{Th}$.
    }
    
    \note{
        Since voltage and current are directly proportional, as seen in Ohm's Law, there is no additional input. Therefore, the corresponding matrix $A$ is a 4 x 1 vector.
    }
    
    \sol{
        According to Ohm's Law, $V = IR$. We are estimating the resistance so $R_{Th}$ corresponds to $\vec{x}$ in the $A\vec{x}=\vec{b}$ Least Squares equation.
        Additionally, $I$ corresponds to $A$ and $V$ corresponds to $\vec{b}$. 
        $$A = \begin{bmatrix} 1 \\ 2 \\ 3 \\ 4 \end{bmatrix}  \vec{b} = \begin{bmatrix} 1.25 \\ 1 \\ 4 \\ 3.5 \end{bmatrix}$$
        By running Least Squares, $R_{Th} = (\mathbf{A}^T\mathbf{A})^{-1}\mathbf{A}^T\vec{b} = 975 kohms$ %how to input ohms
    }

\end{enumerate}